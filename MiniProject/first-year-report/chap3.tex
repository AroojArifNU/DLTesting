% Adjusting chapter title format for regular (numbered) chapters
\titleformat{\chapter}[display]
  {\normalfont\huge\bfseries\centering}{\chaptertitlename\ \thechapter}{20pt}{\Huge}

% Using similar styling for unnumbered chapters but without "Chapter" prefix
\titleformat{name=\chapter,numberless}
  {\normalfont\huge\bfseries\centering}{}{0pt}{\Huge}

\titlespacing*{\chapter}{0pt}{50pt}{40pt} % Adjust vertical spacing before and after the title

\chapter{Proposed Framework} % Ensures chapter numbering starts correctly
\label{chp:3}


This chapter presents a comprehensive approach for AI systems, which may consist of one or multiple DNN components, each component handle different task. My research introduces a framework that addresses each DNN separately, recognising that each has unique specifications, e.g., consider an AI system using the MNIST dataset for different purposes. One DNN might classify handwritten digits (0-9), while another might analyse pairs of digits to predict their sum or product. Although both DNNs work with digit images in this case, their properties in specifications differ due to their distinct tasks, even when using the same dataset. In other cases, such as an autonomous car, different DNNs might use different datasets designed for their specific tasks. One DNN might use labeled images for object detection (e.g., identifying pedestrians, vehicles, and road signs), while another DNN might use data on driving paths and movement for path planning to predict optimal routes. This framework provides end-to-end testing for each DNN. It consists of five components, summarized in Figure~\ref{fig:framework}.

\begin{enumerate}
  \item \emph{Specification} defines key properties to guide testing. It includes details about the DNN architecture, environmental properties, and input data, like data type, sample size, and number of classes. 
    \item \emph{Sampling} involves selecting a representative subset of inputs from a potentially vast input space. Depending on the testing objectives, this subset helps ensure that the tests are meaningful and cover various scenarios effectively.
    \item \emph{Testcase Generation} applies the properties mentioned in the specifications to the sampled data to generate test cases.
    \item \emph{Testing Graph Analysis} conducts both local and global coverage assessments. Locally, it focuses on an individual property to check its correctness. Globally, it examines multiple properties to understand their interdependencies and overall system behavior. By modeling these dependencies probabilistically, testing graph analysis provides a comprehensive view of AI system.
    \item \emph{Error Summarisation} quantify the performance of the AI system by compiling and analyzing recorded errors. This analysis generates graphical reports and recommendations, helping to refine and improve the models.
\end{enumerate}

\begin{figure*}
  \centering
  \includegraphics[width=\linewidth]{figures/fivesteps.png}
  \caption{Overview of the Proposed Framework}
  \label{fig:framework}
\end{figure*}

My research focus on AI systems with DNN components performing various tasks, which may include classification, regression, clustering and more. Each DNN within the system has unique specifications. Formally, we define an AI system $ \mathcal{S} $ as follows:

$\mathcal{F}$ is the functional unit comprises $n$ DNN components $ f_1, \dots, f_n $, each handling different tasks, and a symbolic (software) component $ \omega $ that integrates the outputs of these DNNs. Given an input $\vec{x} = (x_1, \dots, x_n)$, where each $x_i$ represents an input from the respective dataset of $f_i$, the output $\mathcal{F}(\vec{x})$ is defined as $\omega(f_1(x_1), \dots, f_n(x_n))$. Each DNN component $f_i$ processes its input $x_i$ according to its specific dataset.


\section{Specification}
The first component of this framework is formalized to specify model architecture $\mathcal{M}$, environmental properties, testing type, and input data characteristics $\mathcal{D}$. 

The specifications $\mathcal{S}_{\text{specs}}$ include all classes (classification task) by default, but users can adjust this to test specific classes based on their needs.

It is essential to specify the type of testing $\mathcal{T}$ within the $\mathcal{S}_{\text{specs}}$, as different methods are employed based on this choice. In black-box testing $\mathcal{T}_{\text{black}}$, semantic adversarial properties $\mathcal{P}_{\text{sem}}$ are applied by adjusting the input data $\mathcal{D}$ without any knowledge of the model's internal details. In white-box testing $\mathcal{T}_{\text{white}}$, adversarial attacks $\mathcal{P}_{\text{adv}}$ are used by accessing the model's internal structure and parameters. Grey-box testing $\mathcal{T}_{\text{grey}}$ combines aspects of both, utilizing partial knowledge of the model’s structure to inform the testing process. 



\begin{example}
  \label{ex:mnist-adder-specification}
  Consider the AI system $\mathcal{S}_{\text{adder}}$, an \emph{MNIST Digit Adder}, which is specified as $\mathcal{S}_{\text{adder}} = (\mathcal{F}, \mathcal{D})$. In this system, the functional unit $\mathcal{F}$ consists of a DNN component $\mathcal{M}_{\text{CNN}}$, which is a Convolutional Neural Network (CNN) designed to recognize digits ranging from 0 to 9. The dataset $\mathcal{D}$ provides input in the form of images, and a software component $\omega$ is used to perform the addition of the recognized digits. 

  Formally, we define $\mathcal{F} = (\{\mathcal{M}_{\text{CNN}}\}, \omega)$, where $\omega$ integrates the output of the DNN. The function $\mathcal{M}_{\text{CNN}}$ takes single digit as an input and recognizes the digits. The software component $\omega$ can then pick any two recognized digits and compute their sum. For example, $\mathcal{S}_{\text{adder}}$ can perform the task of digit addition:
  \begin{equation}
    \mathcal{S}_{\text{adder}}(x_1, x_2) = \omega(\mathcal{M}_{\text{CNN}}(x_1), \mathcal{M}_{\text{CNN}}(x_2)),
  \end{equation}
  where $\omega(a, b) = a + b$. Given two digits $x_1$ and $x_2$, $\mathcal{S}_{\text{adder}}$ recognizes the digits and computes their sum.

  The specifications $\mathcal{S}_{\text{specs}}$ for this system include the following key elements: The DNN component $\mathcal{M}_{\text{CNN}}$ is a pre-trained CNN with a typical architecture involving a convolutional layer (32 filters, kernel size 3x3, ReLU activation), followed by a max pooling layer (2x2), a flattened layer, a fully connected layer with 128 neurons (ReLU), and an output layer with 10 neurons using softmax activation. The input data $\mathcal{D}$ consists of 10,000 test samples of digits, each labeled with one of 10 classes representing the digits 0 to 9. 

  $\mathcal{S}_{\text{adder}}$ is evaluated using both black-box and white-box testing methods. In black-box testing $\mathcal{T}_{\text{black}}$, semantic adversarial properties such as rotation, brightness adjustment, blur, shear, and contrast modification are applied to the input data without accessing the internal structure of $\mathcal{M}_{\text{CNN}}$. Specifically, $\mathcal{S}_{\text{adder}}$ robustness is tested against rotations within a range of 3 to 30 degrees and brightness adjustments between 0.5 and 1.5. In white-box testing $\mathcal{T}_{\text{white}}$, adversarial examples are generated by utilizing the internal structure and parameters of $\mathcal{M}_{\text{CNN}}$, like Fast Gradient Sign Method (FGSM), Projected Gradient Descent (PGD), AdamPGD, and Deepfool.
\end{example}


\textbf{Note:} Grey-box testing, involving partial model access, will be explored in future work.
\section{Sampling}
The sampling process is designed to identify both efficient and corner cases, focusing on hard-to-learn examples to enhance the robustness of the model. Data is sampled according to the requirements specified in the specifications. By default, all classes are tested, but users can customize the classes to be sampled based on specific requirements. The sampling approach is applied as follows for each classifier $f_i$ independently:

The full sample $\mathcal{X}_i$ for classifier $f_i$, $i=1,\dots,n$, is computed by applying the sampling method directly to the dataset $\mathcal{D}_i$:

\begin{equation}
\mathcal{X}_i = \text{Sampling}(\mathcal{D}_i)
\end{equation}

Here, $\mathcal{D}_i$ represents the dataset associated with each classifier $f_i$. In cases where all classifiers share the same dataset, $\mathcal{D}_i$ may be identical across all $i$, denoted simply as $\mathcal{D}$. However, if the datasets differ for each classifier, they are represented individually as $\mathcal{D}_1, \mathcal{D}_2, \dots, \mathcal{D}_n$ to reflect the specific dataset for each $f_i$.



This sampling method is designed to balance the dataset and ensure that both typical and challenging cases are adequately represented, thereby improving the evaluation of the model's performance across a variety of scenarios.


\begin{example}
  \label{ex:sampling}
  To further illustrate the sampling process, consider the classifier $\mathcal{M}_{\text{CNN}}$ and $\mathcal{D}$ from Example~\ref{ex:mnist-adder-specification}. For each class $c$ in $\mathcal{D}$, we begin with the samples specified in specification, forming the initial subset $\mathcal{X}_{\text{mnist}}$. This subset is then enhanced by applying a hybrid sampling method, that combines Borderline-SMOTE focuses on creating samples near decision boundaries, while ADASYN targets hard-to-learn examples, ensuring that both typical and challenging cases are effectively covered. The final sample $\mathcal{S}_{\text{mnist}}$ for testing is obtained as:
  

  \begin{equation}
    \mathcal{X}_{\text{mnist}} = \text{HybridSampling}(\mathcal{X}_{\text{mnist}})
  \end{equation}

  where the HybridSampling method combines Borderline-SMOTE and ADASYN. This process ensures a balanced and comprehensive set of test samples, covering both typical and challenging cases for robust evaluation of $\mathcal{M}_{\text{CNN}}$.
\end{example}




\begin{tcolorbox}[colback=purple!2!white, colframe=purple, title=Challenges in Sampling]
  \begin{itemize}
    \item \textbf{Synthetic Sample Quality:} Ensuring generated samples represent true corner cases, not noise.
    \item \textbf{Computational Overhead:} Managing the intensive computation required for hybrid sampling.
 
  \end{itemize}

\end{tcolorbox}


\section{Test Case Generation}

The test case generation process aims to create test cases based on the given specifications to evaluate the correctness/robustness of the AI system.

Let $S_i^c$ be the set of samples produced in the sampling step for the classifier $f_i$ and a class $c$. For each perturbation $p\in P$, we generate a set $\T_p^c$ of test cases. Specifically, for each sample $x\in S$ we produce $testcases(p,c,x)$ according to $p$. Then $\T_p^c = \bigcup_{x\in S} testcases(p,c,x)$.

\begin{example}
To generate test cases for the MNIST Digit Adder $\S_{\text{MNIST}}$, we use the specifications defined in Example 2, which include noise and rotation perturbations. Let $S$ be the set of sampled images obtained in Example 3. For each pair of images $(x_1, x_2) \in S$, we define the following test cases:
\begin{itemize}
    \item \textbf{Rotation}: For a given angle $\theta$, generate the perturbed images $x_1' = \text{rotate}(x_1, \theta)$ and $x_2' = \text{rotate}(x_2, \theta)$. The test case is then:
    $\mathcal{T}_{\text{rotation}} = \left\{(x_1', x_2') \mid x_1', x_2' \in \text{rotate}(S, \theta)\right\}$
    \item \textbf{Noise}: For a given mean $\mu$ and standard deviation $\sigma$, generate the perturbed images $x_1'' = \text{noise}(x_1, \mu, \sigma)$ and $x_2'' = \text{noise}(x_2, \mu, \sigma)$. The test case is then:
    $\mathcal{T}_{\text{noise}} = \left\{(x_1'', x_2'') \mid x_1'', x_2'' \in \text{noise}(S, \mu, \sigma)\right\}$
\end{itemize}
The overall set of test cases $\mathcal{T}$ is the union of the individual test cases:
$\mathcal{T} = \mathcal{T}_{\text{rotation}} \cup \mathcal{T}_{\text{noise}}$
\end{example}

\section{Validation}

The validation process aims to evaluate the correctness and robustness of the AI system under various perturbations.

Fix a perturbation \( p \in P \) and a class \( c \). For every test case in \( \mathcal{T}_p^c \), we store the results in the form:
\[ \text{Raw}_{p, c} = \Big\{\big(\text{query}(f_i, x, c), f_i(x)_c\big) \mid x \in \mathcal{T}_p^c \Big\} \]

After generating test cases, measure the AI subsystem's confidence for each class under each type of property.

\subsection{Local Robustness}

Local robustness involves evaluating the AI subsystem's performance on individual images subjected to various transformations. For each image \( x \) from a set of samples \( S_i^c \), the AI subsystem produces a confidence score \( f_i(x) \) representing its certainty in recognizing the class \( c \). The local robustness for a transformation \( T \) applied to an image \( x \) is defined as:

\[ \text{Local Robustness}(x, T) = f_i(T(x)) \]

where \( T \) can be any transformation such as noise addition, rotation, brightness adjustment, occlusion, or scaling. Each transformation is evaluated to determine its impact on the confidence score.

To quantify local robustness, we compute the accuracy for each transformation applied to the images of a specific class:

\[
\text{Local Robustness}(c, T) = \frac{1}{|S_i^c|} \sum_{x \in S_i^c} \mathbb{I}[f_i(T(x)) = c]
\]

where:
- \(\text{Local Robustness}(c, T)\) is the accuracy of the classifier \( f_i \) for class \( c \) under transformation \( T \).
- \( \mathbb{I}[f_i(T(x)) = c] \) is an indicator function that is 1 if the classifier correctly predicts the class \( c \) for the transformed image \( T(x) \), and 0 otherwise.
- \(|S_i^c|\) is the number of correctly classified images in class \( c \).

\begin{example}
Consider the class \( c = 3 \) and the transformation \( T \) being a rotation by 25 degrees. If we have three images \( x_1, x_2, x_3 \) from class \( c \), and the model correctly classifies \( x_1 \) and \( x_2 \) but misclassifies \( x_3 \), the local robustness is computed as follows:

\[
\text{Local Robustness}(3, \text{rotation}) = \frac{1}{3} (\mathbb{I}[f_i(\text{rotate}(x_1, 25)) = 3] + 
\]
\[
\mathbb{I}[f_i(\text{rotate}(x_2, 25)) = 3] + \mathbb{I}[f_i(\text{rotate}(x_3, 25)) = 3])
\]


If the indicator values are 1, 1, and 0 respectively, the local robustness would be:

\[
\text{Local Robustness}(3, \text{rotation}) = \frac{1}{3} (1 + 1 + 0) = \frac{2}{3} \approx 0.67
\]
\end{example}

\subsection{Global Robustness}

Global robustness evaluates the AI subsystem's performance across multiple images and transformations. It is an aggregate measure of how well the AI system performs under various properties on a set of images $S_i^c$.

For a given transformation $T$ and a set of images $S_i^c$, the global robustness is defined as:

$$ \text{Global Robustness}(S_i^c, T) = \frac{1}{|S_i^c|} \sum_{x \in S_i^c} \mathbb{I}[f_i(T(x)) = c] $$

This measures the average confidence score for the AI subsystem over the entire dataset when subjected to a particular transformation.

To quantify global robustness, we compute the expected confidence score over all transformations and images. Depending on whether the relationship is AND or OR, the formulas vary:

\textbf{AND Relationship:}

For an AND relationship between transformations, the global robustness is calculated as:

$$ P(\text{Property 1} \cap \text{Property 2}) = P(\text{Property 1}) \times P(\text{Property 2}) $$

\textbf{OR Relationship:}

For an OR relationship between transformations, the global robustness is calculated as:

$$ P(\text{Property 1} \cup \text{Property 2}) = P(\text{Property 1}) + P(\text{Property 2}) - P(\text{Property 1} \cap \text{Property 2}) $$

\begin{example}
Consider a pair of images $(x_1, x_2)$ representing the digits '3' and '5'. Let the transformation $T$ be rotation. If the model's confidence scores for these transformations are as follows:

$$
\begin{aligned}
&f_i(\text{rotation}(x_1)) = 0.78, \\
&f_i(\text{rotation}(x_2)) = 0.85,
\end{aligned}
$$

For the AND relationship, the global correctness for the pair is computed as follows:

$$P(\text{Global Robustness}_{\text{AND}}) = 0.78 \times 0.85$$

For the OR relationship, the global correctness for the pair is computed as follows:

$$P(\text{Global Robustness}_{\text{OR}}) = 0.78 + 0.85 - (0.78 \times 0.85)$$

(Note: The complete OR relationship formula requires including all images and transformations as per the OR formula.)

\end{example}

\subsection{Use Cases and Examples}
To illustrate the application of ProbLog for global robustness in real-world scenarios, we present the following use cases:

\subsubsection{Use Case 1: Handwritten Digit Recognition}
\textbf{Scenario:} Consider an AI system designed to recognize handwritten digits, such as the MNIST dataset. The system is evaluated under various transformations, including noise addition, rotation, and brightness adjustment. The goal is to determine the global robustness of the system in recognizing digit pairs correctly under these properties.

The tables below provide the probabilities for correctly recognizing digit pairs under different conditions (AND and OR relationships) for an MNIST 2-digit addition system. Each world represents a different combination of transformations applied to the digits.

\begin{table}[h]
    \centering
    \caption{Specification Probabilities for MNIST 2-Digit Addition Under Different Transformations}
    \label{tab:mnist_prob_and_or}
    \resizebox{\textwidth}{!}{%
    \begin{tabular}{|c|c|c|c|c|c|}
    \hline
    \textbf{World} & \textbf{Conditions} & \textbf{Probability Expression (AND)} & \textbf{Probability (AND)} & \textbf{Probability Expression (OR)} & \textbf{Probability (OR)} \\
    \hline
    $w_1$ & \{noise(0), noise(1)\} & $0.85 \cdot 0.8$ & $0.68$ & $0.85 + 0.8 - (0.85 \cdot 0.8)$ & $0.97$ \\
    $w_2$ & \{noise(0), correct(1)\} & $0.85 \cdot 0.9$ & $0.765$ & $0.85 + 0.9 - (0.85 \cdot 0.9)$ & $0.985$ \\
    $w_3$ & \{correct(0), noise(1)\} & $0.9 \cdot 0.8$ & $0.72$ & $0.9 + 0.8 - (0.9 \cdot 0.8)$ & $0.98$ \\
    $w_4$ & \{rotation(0), correct(1)\} & $0.88 \cdot 0.9$ & $0.792$ & $0.88 + 0.9 - (0.88 \cdot 0.9)$ & $0.992$ \\
    $w_5$ & \{correct(0), rotation(1)\} & $0.9 \cdot 0.77$ & $0.693$ & $0.9 + 0.77 - (0.9 \cdot 0.77)$ & $0.977$ \\
    $w_6$ & \{rotation(0), rotation(1)\} & $0.88 \cdot 0.77$ & $0.6776$ & $0.88 + 0.77 - (0.88 \cdot 0.77)$ & $0.9696$ \\
    $w_7$ & \{noise(0), rotation(1)\} & $0.85 \cdot 0.77$ & $0.6545$ & $0.85 + 0.77 - (0.85 \cdot 0.77)$ & $0.9655$ \\
    $w_8$ & \{rotation(0), noise(1)\} & $0.88 \cdot 0.8$ & $0.704$ & $0.88 + 0.8 - (0.88 \cdot 0.8)$ & $0.976$ \\
    $w_9$ & \{correct(0), correct(1)\} & $0.9 \cdot 0.9$ & $0.81$ & $0.9 + 0.9 - (0.9 \cdot 0.9)$ & $0.99$ \\
    \hline
    \end{tabular}
    }
\end{table}

\textbf{Explanation:} The table shows the global correctness probabilities for pairs of digits under various transformation conditions. Each row represents a different combination of transformations applied to the two digits in the pair:
- AND Probability: The probability that both digits are correctly recognized under the specified transformations.
- OR Probability: The probability that at least one of the digits is correctly recognized under the specified transformations.

For example, in world $w_1$, both digits are subjected to noise, leading to an AND probability of $0.68$ and an OR probability of $0.97$.

\textbf{Problog Code:}
\begin{mdframed}[leftline=false, rightline=false, topline=true, bottomline=true]
\scriptsize
\begin{verbatim}
% Define probabilities for digit 0 under different transformations
0.9::noise_0. % Digit 0 correctly predicted with 90% probability under noise
0.85::brightness_0. % Digit 0 correctly predicted with 85% probability under brightness
0.88::rotation_0. % Digit 0 correctly predicted with 88% probability under rotation

% Define probabilities for digit 1 under different transformations
0.8::noise_1. % Digit 1 correctly predicted with 80% probability under noise
0.75::brightness_1. % Digit 1 correctly predicted with 75% probability under brightness
0.77::rotation_1. % Digit 1 correctly predicted with 77% probability under rotation

% Define rules for correct prediction under each transformation for digit 0
correct_noise_0 :- noise_0.
correct_brightness_0 :- brightness_0.
correct_rotation_0 :- rotation_0.

% Define rules for correct prediction under each transformation for digit 1
correct_noise_1 :- noise_1.
correct_brightness_1 :- brightness_1.
correct_rotation_1 :- rotation_1.

% Define rules for incorrect prediction under each transformation for digit 0
wrong_noise_0 :- +correct_noise_0.
wrong_brightness_0 :- +correct_brightness_0.
wrong_rotation_0 :- +correct_rotation_0.

% Define rules for incorrect prediction under each transformation for digit 1
wrong_noise_1 :- +correct_noise_1.
wrong_brightness_1 :- +correct_brightness_1.
wrong_rotation_1 :- +correct_rotation_1.

% Define rules for correct prediction of both digits under noise
pair_correct_noise_0_1 :- correct_noise_0, correct_noise_1.
% Define rules for incorrect prediction of both digits under noise
pair_wrong_noise_0_1 :- wrong_noise_0, wrong_noise_1.

% Define global correctness based on either both correct or both incorrect under noise
global_correct_noise_0_1 :- pair_correct_noise_0_1; pair_wrong_noise_0_1.

% Query the global correctness under noise
query(global_correct_noise_0_1).
\end{verbatim}
\end{mdframed}
\captionof{figure}{Problog code snippet for evaluating handwritten digit recognition under noise, brightness, and rotation transformations.}

\textbf{Explanation:} In this scenario, we are interested in the global correctness of recognizing pairs of digits (0 and 1) under different transformations. The ProbLog code models the local robustness probabilities for each transformation and combines them to evaluate the global correctness.

\subsubsection{Use Case 2: Autonomous Vehicle Perception}

\textbf{Scenario:} An AI system used in autonomous vehicles must reliably detect objects such as vehicles under various weather conditions (rain, sand, fog, and snow). The goal is to evaluate the system's robustness in identifying these objects correctly under these weather conditions.

\begin{mdframed}[leftline=false, rightline=false, topline=true, bottomline=true]
  \scriptsize
  \begin{verbatim}

% Probabilities for Vehicle Detection under Different Weather Conditions
0.75::rain_vehicle. % Vehicle correctly detected with 75% probability under rain
0.55::fog_vehicle. % Vehicle correctly detected with 55% probability under fog
0.7::snow_vehicle. % Vehicle correctly detected with 70% probability under snow

% Correct Detection Rules for Vehicle
correct_rain_vehicle :- rain_vehicle.
correct_fog_vehicle :- fog_vehicle.
correct_snow_vehicle :- snow_vehicle.

% Incorrect Detection Rules for Vehicle
wrong_rain_vehicle :- +correct_rain_vehicle.
wrong_fog_vehicle :- +correct_fog_vehicle.
wrong_snow_vehicle :- +correct_snow_vehicle.

% AND conditions for Vehicle Detection under all weather conditions
global_correct_vehicle_and :- correct_rain_vehicle, correct_fog_vehicle, correct_snow_vehicle.

% OR conditions for Vehicle Detection under any weather condition
global_correct_vehicle_or :- correct_rain_vehicle; correct_fog_vehicle; correct_snow_vehicle.

% Mixed conditions (AND & OR) for Vehicle Detection
global_correct_mixed_vehicle :- correct_rain_vehicle, (correct_fog_vehicle; correct_snow_vehicle).

% Queries for Global Correctness
query(global_correct_vehicle_and).
query(global_correct_vehicle_or).
query(global_correct_mixed_vehicle).
\end{verbatim}
\end{mdframed}
\captionof{figure}{Problog code snippet for evaluating vehicle detection under different weather conditions.}

\textbf{Explanation:} The ProbLog code assesses the global robustness of the AI system in detecting objects (vehicles) under individual and combined weather conditions. This ensures that the system can reliably perform in diverse environmental scenarios.

\begin{table}[h]
  \centering
  \caption{Specification Probabilities (AND) for Vehicle Detection Under Different Weather Conditions \\
  \( P(A \cap B \cap C) = P(A) \times P(B) \times P(C) \)}
  \label{tab:veh_prob_and}
  \resizebox{\textwidth}{!}{%
  \begin{tabular}{|c|c|c|c|}
  \hline
  \textbf{World} & \textbf{Conditions} & \textbf{Probability Expression (AND)} & \textbf{Probability (AND)} \\
  \hline
  $w_1$ & \{rain, fog\} & $0.75 \times 0.55$ & $0.4125$ \\
  $w_2$ & \{rain, snow\} & $0.75 \times 0.7$ & $0.525$ \\
  $w_3$ & \{rain, sand\} & $0.75 \times 0.6$ & $0.45$ \\
  $w_4$ & \{fog, snow\} & $0.55 \times 0.7$ & $0.385$ \\
  $w_5$ & \{fog, sand\} & $0.55 \times 0.6$ & $0.33$ \\
  $w_6$ & \{snow, sand\} & $0.7 \times 0.6$ & $0.42$ \\
  $w_7$ & \{rain, fog, snow\} & $0.75 \times 0.55 \times 0.7$ & $0.28875$ \\
  $w_8$ & \{rain, fog, sand\} & $0.75 \times 0.55 \times 0.6$ & $0.2475$ \\
  $w_9$ & \{rain, snow, sand\} & $0.75 \times 0.7 \times 0.6$ & $0.315$ \\
  $w_{10}$ & \{fog, snow, sand\} & $0.55 \times 0.7 \times 0.6$ & $0.231$ \\
  \hline
  \end{tabular}
  }
\end{table}

\textbf{Explanation:} This table shows the global correctness probabilities for detecting vehicles under different combinations of weather conditions using AND relationships. Each row represents a different combination of weather conditions applied to the detection scenario:
- AND Probability: The probability that the vehicle is correctly detected under all specified weather conditions.

For example, in world $w_1$, the vehicle detection system is subjected to rain and fog, leading to an AND probability of $0.4125$.

\begin{table}[h]
  \centering
  \caption{Specification Probabilities (OR) for Vehicle Detection Under Different Weather Conditions \\
  \( P(A \cup B \cup C) = P(A) + P(B) + P(C) - P(A \cap B) - P(A \cap C) - P(B \cap C) + P(A \cap B \cap C) \)}
  \label{tab:veh_prob_or}
  \resizebox{\textwidth}{!}{%
  \begin{tabular}{|c|c|c|c|}
  \hline
  \textbf{World} & \textbf{Conditions} & \textbf{Probability Expression (OR)} & \textbf{Probability (OR)} \\
  \hline
  $w_1$ & \{rain; fog\} & $0.75 + 0.55 - (0.75 \times 0.55)$ & $0.8875$ \\
  $w_2$ & \{rain; snow\} & $0.75 + 0.7 - (0.75 \times 0.7)$ & $0.925$ \\
  $w_3$ & \{rain; sand\} & $0.75 + 0.6 - (0.75 \times 0.6)$ & $0.9$ \\
  $w_4$ & \{fog; snow\} & $0.55 + 0.7 - (0.55 \times 0.7)$ & $0.835$ \\
  $w_5$ & \{fog; sand\} & $0.55 + 0.6 - (0.55 \times 0.6)$ & $0.82$ \\
  $w_6$ & \{snow; sand\} & $0.7 + 0.6 - (0.7 \times 0.6)$ & $0.88$ \\
  $w_7$ & \{rain; fog; snow\} & $0.75 + 0.55 + 0.7 - (0.75 \times 0.55) - (0.75 \times 0.7) - (0.55 \times 0.7) + (0.75 \times 0.55 \times 0.7)$ & $0.966625$ \\
  $w_8$ & \{rain; fog; sand\} & $0.75 + 0.55 + 0.6 - (0.75 \times 0.55) - (0.75 \times 0.6) - (0.55 \times 0.6) + (0.75 \times 0.55 \times 0.6)$ & $0.95125$ \\
  $w_9$ & \{rain; snow; sand\} & $0.75 + 0.7 + 0.6 - (0.75 \times 0.7) - (0.75 \times 0.6) - (0.7 \times 0.6) + (0.75 \times 0.7 \times 0.6)$ & $0.967$ \\
  $w_{10}$ & \{fog; snow; sand\} & $0.55 + 0.7 + 0.6 - (0.55 \times 0.7) - (0.55 \times 0.6) - (0.7 \times 0.6) + (0.55 \times 0.7 \times 0.6)$ & $0.938$ \\
  \hline
  \end{tabular}
  }
\end{table}

\textbf{Explanation:} This table shows the global correctness probabilities for detecting vehicles under different combinations of weather conditions using OR relationships. Each row represents a different combination of weather conditions applied to the detection scenario:
- OR Probability: The probability that the vehicle is correctly detected under at least one of the specified weather conditions.

For example, in world $w_1$, the vehicle detection system is subjected to rain and fog, leading to an OR probability of $0.8875$.




% \subsection{Use Cases and Examples}

% To illustrate the application of Problog for global robustness in real-world scenarios, we present the following use cases:

% \subsection{Use Case 1: Handwritten Digit Recognition}

% \textbf{Scenario:} Consider an AI system designed to recognize handwritten digits, such as the MNIST dataset. The system is evaluated under various transformations, including noise addition, rotation, and brightness adjustment. The goal is to determine the global robustness of the system in recognizing digit pairs correctly under these properties.


  
%   \begin{table}[h]
%     \centering
  
%     \caption{Specification Probabilities for MNIST 2-Digit Addition Under Different Transformations}
%     \label{tab:mnist_prob_and_or}
  
%     \resizebox{\textwidth}{!}{%
%     \begin{tabular}{|c|c|c|c|c|c|}
%     \hline
%     \textbf{World} & \textbf{Conditions} & \textbf{Probability Expression (AND)} & \textbf{Probability (AND)} & \textbf{Probability Expression (OR)} & \textbf{Probability (OR)} \\
%     \hline
%     $w_1$ & \{noise(0), noise(1)\} & $0.85 \cdot 0.8$ & $0.68$ & $0.85 + 0.8 - (0.85 \cdot 0.8)$ & $0.97$ \\
%     $w_2$ & \{noise(0), correct(1)\} & $0.85 \cdot 0.9$ & $0.765$ & $0.85 + 0.9 - (0.85 \cdot 0.9)$ & $0.985$ \\
%     $w_3$ & \{correct(0), noise(1)\} & $0.9 \cdot 0.8$ & $0.72$ & $0.9 + 0.8 - (0.9 \cdot 0.8)$ & $0.98$ \\
%     $w_4$ & \{rotation(0), correct(1)\} & $0.88 \cdot 0.9$ & $0.792$ & $0.88 + 0.9 - (0.88 \cdot 0.9)$ & $0.992$ \\
%     $w_5$ & \{correct(0), rotation(1)\} & $0.9 \cdot 0.77$ & $0.693$ & $0.9 + 0.77 - (0.9 \cdot 0.77)$ & $0.977$ \\
%     $w_6$ & \{rotation(0), rotation(1)\} & $0.88 \cdot 0.77$ & $0.6776$ & $0.88 + 0.77 - (0.88 \cdot 0.77)$ & $0.9696$ \\
%     $w_7$ & \{noise(0), rotation(1)\} & $0.85 \cdot 0.77$ & $0.6545$ & $0.85 + 0.77 - (0.85 \cdot 0.77)$ & $0.9655$ \\
%     $w_8$ & \{rotation(0), noise(1)\} & $0.88 \cdot 0.8$ & $0.704$ & $0.88 + 0.8 - (0.88 \cdot 0.8)$ & $0.976$ \\
%     $w_9$ & \{correct(0), correct(1)\} & $0.9 \cdot 0.9$ & $0.81$ & $0.9 + 0.9 - (0.9 \cdot 0.9)$ & $0.99$ \\
%     \hline
%     \end{tabular}
%     }
%     \end{table}
  



% \textbf{Problog Code:}
% \begin{mdframed}[leftline=false, rightline=false, topline=true, bottomline=true]
% \scriptsize
% \begin{verbatim}
% % Define probabilities for digit 0 under different transformations
% 0.9::noise_0. % Digit 0 correctly predicted with 90% probability under noise
% 0.85::brightness_0. % Digit 0 correctly predicted with 85% probability under brightness
% 0.88::rotation_0. % Digit 0 correctly predicted with 88% probability under rotation

% % Define probabilities for digit 1 under different transformations
% 0.8::noise_1. % Digit 1 correctly predicted with 80% probability under noise
% 0.75::brightness_1. % Digit 1 correctly predicted with 75% probability under brightness
% 0.77::rotation_1. % Digit 1 correctly predicted with 77% probability under rotation

% % Define rules for correct prediction under each transformation for digit 0
% correct_noise_0 :- noise_0.
% correct_brightness_0 :- brightness_0.
% correct_rotation_0 :- rotation_0.

% % Define rules for correct prediction under each transformation for digit 1
% correct_noise_1 :- noise_1.
% correct_brightness_1 :- brightness_1.
% correct_rotation_1 :- rotation_1.

% % Define rules for incorrect prediction under each transformation for digit 0
% wrong_noise_0 :- +correct_noise_0.
% wrong_brightness_0 :- +correct_brightness_0.
% wrong_rotation_0 :- +correct_rotation_0.

% % Define rules for incorrect prediction under each transformation for digit 1
% wrong_noise_1 :- +correct_noise_1.
% wrong_brightness_1 :- +correct_brightness_1.
% wrong_rotation_1 :- +correct_rotation_1.

% % Define rules for correct prediction of both digits under noise
% pair_correct_noise_0_1 :- correct_noise_0, correct_noise_1.
% % Define rules for incorrect prediction of both digits under noise
% pair_wrong_noise_0_1 :- wrong_noise_0, wrong_noise_1.

% % Define global correctness based on either both correct or both incorrect under noise
% global_correct_noise_0_1 :- pair_correct_noise_0_1; pair_wrong_noise_0_1.

% % Query the global correctness under noise
% query(global_correct_noise_0_1).
% \end{verbatim}
% \end{mdframed}
% \captionof{figure}{Problog code snippet for evaluating handwritten digit recognition under noise, brightness, and rotation transformations.}



% \textbf{Explanation:} In this scenario, we are interested in the global correctness of recognizing pairs of digits (0 and 1) under different transformations. The Problog code models the local robustness probabilities for each transformation and combines them to evaluate the global correctness.


%   \subsection{Use Case 2: Autonomous Vehicle Perception}

% \textbf{Scenario:} An AI system used in autonomous vehicles must reliably detect objects such as pedestrians and vehicles under various weather conditions (rain, fog, and snow). The goal is to evaluate the system's robustness in identifying these objects correctly under these weather conditions.



% \begin{mdframed}[leftline=false, rightline=false, topline=true, bottomline=true]
%   \scriptsize
%   \begin{verbatim}

%   % Probabilities for Vehicle Detection under Different Weather Conditions
%   0.75::rain_vehicle. % Vehicle correctly detected with 75% probability under rain
%   0.55::fog_vehicle. % Vehicle correctly detected with 55% probability under fog
%   0.7::snow_vehicle. % Vehicle correctly detected with 70% probability under snow

  
%   % Correct Detection Rules for Vehicle
%   correct_rain_vehicle :- rain_vehicle.
%   correct_fog_vehicle :- fog_vehicle.
%   correct_snow_vehicle :- snow_vehicle.
  
%   % Incorrect Detection Rules for Vehicle
%   wrong_rain_vehicle :- +correct_rain_vehicle.
%   wrong_fog_vehicle :- +correct_fog_vehicle.
%   wrong_snow_vehicle :- +correct_snow_vehicle.
  

%   % AND conditions for Vehicle Detection under all weather conditions
%   global_correct_vehicle_and :- correct_rain_vehicle, correct_fog_vehicle, correct_snow_vehicle.
  
%   % OR conditions for Vehicle Detection under any weather condition
%   global_correct_vehicle_or :- correct_rain_vehicle; correct_fog_vehicle; correct_snow_vehicle.
  
%   % Mixed conditions (AND & OR) for Vehicle Detection
%   global_correct_mixed_vehicle :- correct_rain_vehicle, (correct_fog_vehicle; correct_snow_vehicle).
  
%   % Queries for Global Correctness

%   query(global_correct_vehicle_and).
%   query(global_correct_vehicle_or).
%   query(global_correct_mixed_vehicle).
%   \end{verbatim}
%   \end{mdframed}
%   \captionof{figure}{Problog code snippet for evaluating  vehicle detection under different weather conditions.}
  
  

% \textbf{Explanation:} he Problog code assesses the global robustness of the AI system in detecting objects (pedestrians and vehicles) under individual and combined weather conditions. This ensures that the system can reliably perform in diverse environmental scenarios.

% \begin{table}[h]
%   \centering
%   \caption{Specification Probabilities (AND) for Vehicle Detection Under Different Weather Conditions \\
%   \( P(A \cap B \cap C) = P(A) \times P(B) \times P(C) \)}
%   \label{tab:veh_prob_and}
%   \resizebox{\textwidth}{!}{%
%   \begin{tabular}{|c|c|c|c|}
%   \hline
% \textbf{World} & \textbf{Conditions} & \textbf{Probability Expression (AND)} & \textbf{Probability (AND)} \\
%   \hline
%   $w_1$ & \{rain, fog\} & $0.75 \times 0.55$ & $0.4125$ \\
%   $w_2$ & \{rain, snow\} & $0.75 \times 0.7$ & $0.525$ \\
%   $w_3$ & \{rain, sand\} & $0.75 \times 0.6$ & $0.45$ \\
%   $w_4$ & \{fog, snow\} & $0.55 \times 0.7$ & $0.385$ \\
%   $w_5$ & \{fog, sand\} & $0.55 \times 0.6$ & $0.33$ \\
%   $w_6$ & \{snow, sand\} & $0.7 \times 0.6$ & $0.42$ \\
%   $w_7$ & \{rain, fog, snow\} & $0.75 \times 0.55 \times 0.7$ & $0.28875$ \\
%   $w_8$ & \{rain, fog, sand\} & $0.75 \times 0.55 \times 0.6$ & $0.2475$ \\
%   $w_9$ & \{rain, snow, sand\} & $0.75 \times 0.7 \times 0.6$ & $0.315$ \\
%   $w_{10}$ & \{fog, snow, sand\} & $0.55 \times 0.7 \times 0.6$ & $0.231$ \\
%   \hline
%   \end{tabular}
%   }
% \end{table}


% \begin{table}[h]
%   \centering
%   \caption{Specification Probabilities (OR) for Vehicle Detection Under Different Weather Conditions \\
%   \( P(A \cup B \cup C) = P(A) + P(B) + P(C) - P(A \cap B) - P(A \cap C) - P(B \cap C) + P(A \cap B \cap C) \)}
%   \label{tab:veh_prob_or}
%   \resizebox{\textwidth}{!}{%
%   \begin{tabular}{|c|c|c|c|}
%   \hline
% \textbf{World} & \textbf{Conditions} & \textbf{Probability Expression (OR)} & \textbf{Probability (OR)} \\
%   \hline
%   $w_1$ & \{rain; fog\} & $0.75 + 0.55 - (0.75 \times 0.55)$ & $0.8875$ \\
%   $w_2$ & \{rain; snow\} & $0.75 + 0.7 - (0.75 \times 0.7)$ & $0.925$ \\
%   $w_3$ & \{rain; sand\} & $0.75 + 0.6 - (0.75 \times 0.6)$ & $0.9$ \\
%   $w_4$ & \{fog; snow\} & $0.55 + 0.7 - (0.55 \times 0.7)$ & $0.835$ \\
%   $w_5$ & \{fog; sand\} & $0.55 + 0.6 - (0.55 \times 0.6)$ & $0.82$ \\
%   $w_6$ & \{snow; sand\} & $0.7 + 0.6 - (0.7 \times 0.6)$ & $0.88$ \\
%   $w_7$ & \{rain; fog; snow\} & $0.75 + 0.55 + 0.7 - (0.75 \times 0.55) - (0.75 \times 0.7) - (0.55 \times 0.7) + (0.75 \times 0.55 \times 0.7)$ & $0.966625$ \\
%   $w_8$ & \{rain; fog; sand\} & $0.75 + 0.55 + 0.6 - (0.75 \times 0.55) - (0.75 \times 0.6) - (0.55 \times 0.6) + (0.75 \times 0.55 \times 0.6)$ & $0.95125$ \\
%   $w_9$ & \{rain; snow; sand\} & $0.75 + 0.7 + 0.6 - (0.75 \times 0.7) - (0.75 \times 0.6) - (0.7 \times 0.6) + (0.75 \times 0.7 \times 0.6)$ & $0.967$ \\
%   $w_{10}$ & \{fog; snow; sand\} & $0.55 + 0.7 + 0.6 - (0.55 \times 0.7) - (0.55 \times 0.6) - (0.7 \times 0.6) + (0.55 \times 0.7 \times 0.6)$ & $0.938$ \\
%   \hline
%   \end{tabular}
%   }
% \end{table}

% \section{Error Summarization}

% Error summarization involves evaluating the performance of the AI system by identifying and quantifying errors. We use Bayesian Network-based Coverage Metrics to assess both local and global coverage. Two testing coverage metrics are defined in Figure~\ref{Coverage}: local coverage (LC) and global coverage (GC).

% \begin{figure*}[h]
%     \centering
%     \includegraphics[width=1\textwidth]{figures/step5.pdf}
%     \caption{Error Summarization}
%     \label{Summarization}
% \end{figure*}

% \subsection{Use Cases and Examples}

% To illustrate the application of ProbLog for global robustness in real-world scenarios, we present the following use cases:

% \subsubsection{Use Case 1: Handwritten Digit Recognition}

% \textbf{Scenario:} Consider an AI system designed to recognize handwritten digits, such as the MNIST dataset. The system is evaluated under various transformations, including noise addition, rotation, and brightness adjustment. The goal is to determine the global robustness of the system in recognizing digit pairs correctly under these properties.

% The tables below provide the probabilities for correctly recognizing digit pairs under different conditions (AND and OR relationships) for an MNIST 2-digit addition system. Each world represents a different combination of transformations applied to the digits.

% \begin{table}[h]
%     \centering
%     \caption{Specification Probabilities for MNIST 2-Digit Addition Under Different Transformations}
%     \label{tab:mnist_prob_and_or}
%     \resizebox{\textwidth}{!}{%
%     \begin{tabular}{|c|c|c|c|c|c|}
%     \hline
%     \textbf{World} & \textbf{Conditions} & \textbf{Probability Expression (AND)} & \textbf{Probability (AND)} & \textbf{Probability Expression (OR)} & \textbf{Probability (OR)} \\
%     \hline
%     $w_1$ & \{noise(0), noise(1)\} & $0.85 \cdot 0.8$ & $0.68$ & $0.85 + 0.8 - (0.85 \cdot 0.8)$ & $0.97$ \\
%     $w_2$ & \{noise(0), correct(1)\} & $0.85 \cdot 0.9$ & $0.765$ & $0.85 + 0.9 - (0.85 \cdot 0.9)$ & $0.985$ \\
%     $w_3$ & \{correct(0), noise(1)\} & $0.9 \cdot 0.8$ & $0.72$ & $0.9 + 0.8 - (0.9 \cdot 0.8)$ & $0.98$ \\
%     $w_4$ & \{rotation(0), correct(1)\} & $0.88 \cdot 0.9$ & $0.792$ & $0.88 + 0.9 - (0.88 \cdot 0.9)$ & $0.992$ \\
%     $w_5$ & \{correct(0), rotation(1)\} & $0.9 \cdot 0.77$ & $0.693$ & $0.9 + 0.77 - (0.9 \cdot 0.77)$ & $0.977$ \\
%     $w_6$ & \{rotation(0), rotation(1)\} & $0.88 \cdot 0.77$ & $0.6776$ & $0.88 + 0.77 - (0.88 \cdot 0.77)$ & $0.9696$ \\
%     $w_7$ & \{noise(0), rotation(1)\} & $0.85 \cdot 0.77$ & $0.6545$ & $0.85 + 0.77 - (0.85 \cdot 0.77)$ & $0.9655$ \\
%     $w_8$ & \{rotation(0), noise(1)\} & $0.88 \cdot 0.8$ & $0.704$ & $0.88 + 0.8 - (0.88 \cdot 0.8)$ & $0.976$ \\
%     $w_9$ & \{correct(0), correct(1)\} & $0.9 \cdot 0.9$ & $0.81$ & $0.9 + 0.9 - (0.9 \cdot 0.9)$ & $0.99$ \\
%     \hline
%     \end{tabular}
%     }
% \end{table}

% \textbf{Explanation:} The table shows the global correctness probabilities for pairs of digits under various transformation conditions. Each row represents a different combination of transformations applied to the two digits in the pair:
% - AND Probability: The probability that both digits are correctly recognized under the specified transformations.
% - OR Probability: The probability that at least one of the digits is correctly recognized under the specified transformations.

% For example, in world $w_1$, both digits are subjected to noise, leading to an AND probability of $0.68$ and an OR probability of $0.97$.

% \textbf{Problog Code:}
% \begin{mdframed}[leftline=false, rightline=false, topline=true, bottomline=true]
% \scriptsize
% \begin{verbatim}
% % Define probabilities for digit 0 under different transformations
% 0.9::noise_0. % Digit 0 correctly predicted with 90% probability under noise
% 0.85::brightness_0. % Digit 0 correctly predicted with 85% probability under brightness
% 0.88::rotation_0. % Digit 0 correctly predicted with 88% probability under rotation

% % Define probabilities for digit 1 under different transformations
% 0.8::noise_1. % Digit 1 correctly predicted with 80% probability under noise
% 0.75::brightness_1. % Digit 1 correctly predicted with 75% probability under brightness
% 0.77::rotation_1. % Digit 1 correctly predicted with 77% probability under rotation

% % Define rules for correct prediction under each transformation for digit 0
% correct_noise_0 :- noise_0.
% correct_brightness_0 :- brightness_0.
% correct_rotation_0 :- rotation_0.

% % Define rules for correct prediction under each transformation for digit 1
% correct_noise_1 :- noise_1.
% correct_brightness_1 :- brightness_1.
% correct_rotation_1 :- rotation_1.

% % Define rules for incorrect prediction under each transformation for digit 0
% wrong_noise_0 :- +correct_noise_0.
% wrong_brightness_0 :- +correct_brightness_0.
% wrong_rotation_0 :- +correct_rotation_0.

% % Define rules for incorrect prediction under each transformation for digit 1
% wrong_noise_1 :- +correct_noise_1.
% wrong_brightness_1 :- +correct_brightness_1.
% wrong_rotation_1 :- +correct_rotation_1.

% % Define rules for correct prediction of both digits under noise
% pair_correct_noise_0_1 :- correct_noise_0, correct_noise_1.
% % Define rules for incorrect prediction of both digits under noise
% pair_wrong_noise_0_1 :- wrong_noise_0, wrong_noise_1.

% % Define global correctness based on either both correct or both incorrect under noise
% global_correct_noise_0_1 :- pair_correct_noise_0_1; pair_wrong_noise_0_1.

% % Query the global correctness under noise
% query(global_correct_noise_0_1).
% \end{verbatim}
% \end{mdframed}
% \captionof{figure}{Problog code snippet for evaluating handwritten digit recognition under noise, brightness, and rotation transformations.}

% \textbf{Explanation:} In this scenario, we are interested in the global correctness of recognizing pairs of digits (0 and 1) under different transformations. The ProbLog code models the local robustness probabilities for each transformation and combines them to evaluate the global correctness.

% \subsubsection{Use Case 2: Autonomous Vehicle Perception}

% \textbf{Scenario:} An AI system used in autonomous vehicles must reliably detect objects such as pedestrians and vehicles under various weather conditions (rain, fog, and snow). The goal is to evaluate the system's robustness in identifying these objects correctly under these weather conditions.

% \begin{mdframed}[leftline=false, rightline=false, topline=true, bottomline=true]
%   \scriptsize
%   \begin{verbatim}

% % Probabilities for Vehicle Detection under Different Weather Conditions
% 0.75::rain_vehicle. % Vehicle correctly detected with 75% probability under rain
% 0.55::fog_vehicle. % Vehicle correctly detected with 55% probability under fog
% 0.7::snow_vehicle. % Vehicle correctly detected with 70% probability under snow

% % Correct Detection Rules for Vehicle
% correct_rain_vehicle :- rain_vehicle.
% correct_fog_vehicle :- fog_vehicle.
% correct_snow_vehicle :- snow_vehicle.

% % Incorrect Detection Rules for Vehicle
% wrong_rain_vehicle :- +correct_rain_vehicle.
% wrong_fog_vehicle :- +correct_fog_vehicle.
% wrong_snow_vehicle :- +correct_snow_vehicle.

% % AND conditions for Vehicle Detection under all weather conditions
% global_correct_vehicle_and :- correct_rain_vehicle, correct_fog_vehicle, correct_snow_vehicle.

% % OR conditions for Vehicle Detection under any weather condition
% global_correct_vehicle_or :- correct_rain_vehicle; correct_fog_vehicle; correct_snow_vehicle.

% % Mixed conditions (AND & OR) for Vehicle Detection
% global_correct_mixed_vehicle :- correct_rain_vehicle, (correct_fog_vehicle; correct_snow_vehicle).

% % Queries for Global Correctness
% query(global_correct_vehicle_and).
% query(global_correct_vehicle_or).
% query(global_correct_mixed_vehicle).
% \end{verbatim}
% \end{mdframed}
% \captionof{figure}{Problog code snippet for evaluating vehicle detection under different weather conditions.}

% \textbf{Explanation:} The ProbLog code assesses the global robustness of the AI system in detecting objects (vehicles) under individual and combined weather conditions. This ensures that the system can reliably perform in diverse environmental scenarios.

% \begin{table}[h]
%   \centering
%   \caption{Specification Probabilities (AND) for Vehicle Detection Under Different Weather Conditions \\
%   \( P(A \cap B \cap C) = P(A) \times P(B) \times P(C) \)}
%   \label{tab:veh_prob_and}
%   \resizebox{\textwidth}{!}{%
%   \begin{tabular}{|c|c|c|c|}
%   \hline
%   \textbf{World} & \textbf{Conditions} & \textbf{Probability Expression (AND)} & \textbf{Probability (AND)} \\
%   \hline
%   $w_1$ & \{rain, fog\} & $0.75 \times 0.55$ & $0.4125$ \\
%   $w_2$ & \{rain, snow\} & $0.75 \times 0.7$ & $0.525$ \\
%   $w_3$ & \{rain, sand\} & $0.75 \times 0.6$ & $0.45$ \\
%   $w_4$ & \{fog, snow\} & $0.55 \times 0.7$ & $0.385$ \\
%   $w_5$ & \{fog, sand\} & $0.55 \times 0.6$ & $0.33$ \\
%   $w_6$ & \{snow, sand\} & $0.7 \times 0.6$ & $0.42$ \\
%   $w_7$ & \{rain, fog, snow\} & $0.75 \times 0.55 \times 0.7$ & $0.28875$ \\
%   $w_8$ & \{rain, fog, sand\} & $0.75 \times 0.55 \times 0.6$ & $0.2475$ \\
%   $w_9$ & \{rain, snow, sand\} & $0.75 \times 0.7 \times 0.6$ & $0.315$ \\
%   $w_{10}$ & \{fog, snow, sand\} & $0.55 \times 0.7 \times 0.6$ & $0.231$ \\
%   \hline
%   \end{tabular}
%   }
% \end{table}

% \textbf{Explanation:} This table shows the global correctness probabilities for detecting vehicles under different combinations of weather conditions using AND relationships. Each row represents a different combination of weather conditions applied to the detection scenario:
% - AND Probability: The probability that the vehicle is correctly detected under all specified weather conditions.

% For example, in world $w_1$, the vehicle detection system is subjected to rain and fog, leading to an AND probability of $0.4125$.

% \begin{table}[h]
%   \centering
%   \caption{Specification Probabilities (OR) for Vehicle Detection Under Different Weather Conditions \\
%   \( P(A \cup B \cup C) = P(A) + P(B) + P(C) - P(A \cap B) - P(A \cap C) - P(B \cap C) + P(A \cap B \cap C) \)}
%   \label{tab:veh_prob_or}
%   \resizebox{\textwidth}{!}{%
%   \begin{tabular}{|c|c|c|c|}
%   \hline
%   \textbf{World} & \textbf{Conditions} & \textbf{Probability Expression (OR)} & \textbf{Probability (OR)} \\
%   \hline
%   $w_1$ & \{rain; fog\} & $0.75 + 0.55 - (0.75 \times 0.55)$ & $0.8875$ \\
%   $w_2$ & \{rain; snow\} & $0.75 + 0.7 - (0.75 \times 0.7)$ & $0.925$ \\
%   $w_3$ & \{rain; sand\} & $0.75 + 0.6 - (0.75 \times 0.6)$ & $0.9$ \\
%   $w_4$ & \{fog; snow\} & $0.55 + 0.7 - (0.55 \times 0.7)$ & $0.835$ \\
%   $w_5$ & \{fog; sand\} & $0.55 + 0.6 - (0.55 \times 0.6)$ & $0.82$ \\
%   $w_6$ & \{snow; sand\} & $0.7 + 0.6 - (0.7 \times 0.6)$ & $0.88$ \\
%   $w_7$ & \{rain; fog; snow\} & $0.75 + 0.55 + 0.7 - (0.75 \times 0.55) - (0.75 \times 0.7) - (0.55 \times 0.7) + (0.75 \times 0.55 \times 0.7)$ & $0.966625$ \\
%   $w_8$ & \{rain; fog; sand\} & $0.75 + 0.55 + 0.6 - (0.75 \times 0.55) - (0.75 \times 0.6) - (0.55 \times 0.6) + (0.75 \times 0.55 \times 0.6)$ & $0.95125$ \\
%   $w_9$ & \{rain; snow; sand\} & $0.75 + 0.7 + 0.6 - (0.75 \times 0.7) - (0.75 \times 0.6) - (0.7 \times 0.6) + (0.75 \times 0.7 \times 0.6)$ & $0.967$ \\
%   $w_{10}$ & \{fog; snow; sand\} & $0.55 + 0.7 + 0.6 - (0.55 \times 0.7) - (0.55 \times 0.6) - (0.7 \times 0.6) + (0.55 \times 0.7 \times 0.6)$ & $0.938$ \\
%   \hline
%   \end{tabular}
%   }
% \end{table}

% \textbf{Explanation:} This table shows the global correctness probabilities for detecting vehicles under different combinations of weather conditions using OR relationships. Each row represents a different combination of weather conditions applied to the detection scenario:
% - OR Probability: The probability that the vehicle is correctly detected under at least one of the specified weather conditions.

% For example, in world $w_1$, the vehicle detection system is subjected to rain and fog, leading to an OR probability of $0.8875$.

% \section{Error Summarization}

% Error summarization involves evaluating the performance of the AI system by identifying and quantifying errors. We use Bayesian Network-based Coverage Metrics to assess both local and global coverage. Two testing coverage metrics are defined in Figure~\ref{Coverage}: local coverage (LC) and global coverage (GC).

% \begin{figure*}[h]
%     \centering
%     \includegraphics[width=\textwidth]{figures/step5.pdf}
%     \caption{Error Summarization}
%     \label{Summarization}
% \end{figure*}
\section{Error Summarization}

Error summarization involves evaluating the performance of the AI system by identifying and quantifying errors. We use Bayesian Network-based Coverage Metrics to assess both local and global coverage. Two testing coverage metrics are defined in Figure~\ref{Coverage}: local coverage (LC) and global coverage (GC).

\begin{figure*}[h]
    \centering
    \includegraphics[width=\textwidth]{figures/step5.pdf}
    \caption{Error Summarization}
    \label{Summarization}
\end{figure*}



\section{Algorithm for Evaluating Model Robustness Using ProbLog}

\begin{algorithm}
  \caption{Evaluating Model Robustness Using ProbLog}
  \label{alg:model_robustness}
  \begin{algorithmic}[1]

  \REQUIRE $M$: Pre-trained model
  \REQUIRE $D$: Dataset
  \REQUIRE $N$: Number of samples per class
  \REQUIRE $P$: Set of properties

  \ENSURE $G$: Global correctness values for each specification

  \STATE \textbf{Load and Preprocess Data}
  \STATE $(X_{\text{train}}, y_{\text{train}}), (X_{\text{test}}, y_{\text{test}}) \leftarrow \text{load\_data}(D)$
  \STATE $X_{\text{test}} \leftarrow \text{preprocess\_data}(X_{\text{test}})$
  \STATE $\hat{Y} \leftarrow M.\text{predict}(X_{\text{test}})$
  \STATE $\hat{y} \leftarrow \text{extract\_predictions}(\hat{Y})$

  \STATE \textbf{Determine Number of Classes}
  \STATE $C \leftarrow \text{num\_classes}(y_{\text{test}})$ \COMMENT{Number of unique classes in the dataset}

  \STATE \textbf{Select Correctly Classified Samples}
  \STATE $\text{correct\_samples} \leftarrow \{ \}$
  \FOR{$c = 0$ \TO $C-1$} 
      \STATE $\text{correct\_samples}[c] \leftarrow \text{select\_correct\_samples}(X_{\text{test}}, y_{\text{test}}, \hat{y}, c, N)$
  \ENDFOR

  \STATE \textbf{Define Transformation Functions}
  \STATE $\text{define\_transformations}()$

  \STATE \textbf{Generate Test Cases}
  \FOR{$c = 0$ \TO $C-1$}
      \FORALL{$x \in \text{correct\_samples}[c]$}
          \FORALL{$T \in P$}
              \STATE $\text{generate\_test\_cases}(x, T)$
          \ENDFOR
      \ENDFOR
  \ENDFOR

  \STATE \textbf{Evaluate Model on Transformations}
  \FOR{$c = 0$ \TO $C-1$}
      \FORALL{$\text{property} \in P$}
          \STATE $L_{c, \text{property}} \leftarrow \text{compute\_accuracy}(M, \text{property}, c)$
      \ENDFOR
  \ENDFOR

  \STATE \textbf{Specification Definitions:}
  \STATE Specification1: $P(\text{Property1} \cap \text{Property2}) = P(\text{Property1}) \times P(\text{Property2})$ \COMMENT{AND relationship}
  \STATE Specification2: $P(\text{Property1} \cup \text{Property2}) = P(\text{Property1}) + P(\text{Property2}) - P(\text{Property1} \cap \text{Property2})$ \COMMENT{OR relationship}
  \STATE Specification3: Custom definitions
  \STATE \ldots

  \STATE \textbf{Generate and Evaluate ProbLog Code for Each Specification}
  \STATE Initialize $G$ as an empty list

  \FOR{$\text{spec} = 1$ \TO $\text{num\_specs}$}
      \STATE $\text{ProbLog Code}_{\text{spec}} \leftarrow \text{generate\_problog\_code}(L_{c, \text{property}}, \text{spec})$
      \STATE $G_{\text{spec}} \leftarrow \text{evaluate\_problog}(\text{ProbLog Code}_{\text{spec}})$
      \STATE Append $G_{\text{spec}}$ to $G$
  \ENDFOR

  \RETURN $G$
  \end{algorithmic}
\end{algorithm}

