Adjusting chapter title format for regular (numbered) chapters
\titleformat{\chapter}[display]
  {\normalfont\huge\bfseries\centering}{\chaptertitlename\ \thechapter}{20pt}{\Huge}

% Using similar styling for unnumbered chapters but without "Chapter" prefix
\titleformat{name=\chapter,numberless}
  {\normalfont\huge\bfseries\centering}{}{0pt}{\Huge}

\titlespacing*{\chapter}{0pt}{50pt}{40pt} % Adjust vertical spacing before and after the title

\appendix
\chapter{Useacases}
\label{code}

\section{Use Cases and Examples}
To illustrate the application of ProbLog for global robustness in real-world scenarios, we present the following use cases:

\subsection{Use Case 1: Handwritten Digit Recognition}
\textbf{Scenario:} Consider an AI system designed to recognize handwritten digits, such as the MNIST dataset. The system is evaluated under various transformations, including noise addition, rotation, and brightness adjustment. The goal is to determine the global robustness of the system in recognizing digit pairs correctly under these properties.

The tables below provide the probabilities for correctly recognizing digit pairs under different conditions (AND and OR relationships) for an MNIST 2-digit addition system. Each world represents a different combination of transformations applied to the digits.

\begin{table}[h]
    \centering
    \caption{Specification Probabilities for MNIST 2-Digit Addition Under Different Transformations}
    \label{tab:mnist_prob_and_or}
    \resizebox{\textwidth}{!}{%
    \begin{tabular}{|c|c|c|c|c|c|}
    \hline
    \textbf{World} & \textbf{Conditions} & \textbf{Probability Expression (AND)} & \textbf{Probability (AND)} & \textbf{Probability Expression (OR)} & \textbf{Probability (OR)} \\
    \hline
    $w_1$ & \{noise(0), noise(1)\} & $0.85 \cdot 0.8$ & $0.68$ & $0.85 + 0.8 - (0.85 \cdot 0.8)$ & $0.97$ \\
    $w_2$ & \{noise(0), correct(1)\} & $0.85 \cdot 0.9$ & $0.765$ & $0.85 + 0.9 - (0.85 \cdot 0.9)$ & $0.985$ \\
    $w_3$ & \{correct(0), noise(1)\} & $0.9 \cdot 0.8$ & $0.72$ & $0.9 + 0.8 - (0.9 \cdot 0.8)$ & $0.98$ \\
    $w_4$ & \{rotation(0), correct(1)\} & $0.88 \cdot 0.9$ & $0.792$ & $0.88 + 0.9 - (0.88 \cdot 0.9)$ & $0.992$ \\
    $w_5$ & \{correct(0), rotation(1)\} & $0.9 \cdot 0.77$ & $0.693$ & $0.9 + 0.77 - (0.9 \cdot 0.77)$ & $0.977$ \\
    $w_6$ & \{rotation(0), rotation(1)\} & $0.88 \cdot 0.77$ & $0.6776$ & $0.88 + 0.77 - (0.88 \cdot 0.77)$ & $0.9696$ \\
    $w_7$ & \{noise(0), rotation(1)\} & $0.85 \cdot 0.77$ & $0.6545$ & $0.85 + 0.77 - (0.85 \cdot 0.77)$ & $0.9655$ \\
    $w_8$ & \{rotation(0), noise(1)\} & $0.88 \cdot 0.8$ & $0.704$ & $0.88 + 0.8 - (0.88 \cdot 0.8)$ & $0.976$ \\
    $w_9$ & \{correct(0), correct(1)\} & $0.9 \cdot 0.9$ & $0.81$ & $0.9 + 0.9 - (0.9 \cdot 0.9)$ & $0.99$ \\
    \hline
    \end{tabular}
    }
\end{table}

\textbf{Explanation:} The table shows the global correctness probabilities for pairs of digits under various transformation conditions. Each row represents a different combination of transformations applied to the two digits in the pair:
- AND Probability: The probability that both digits are correctly recognized under the specified transformations.
- OR Probability: The probability that at least one of the digits is correctly recognized under the specified transformations.

For example, in world $w_1$, both digits are subjected to noise, leading to an AND probability of $0.68$ and an OR probability of $0.97$.

\textbf{Problog Code:}
\begin{mdframed}[leftline=false, rightline=false, topline=true, bottomline=true]
\scriptsize
\begin{verbatim}
% Define probabilities for digit 0 under different transformations
0.9::noise_0. % Digit 0 correctly predicted with 90% probability under noise
0.85::brightness_0. % Digit 0 correctly predicted with 85% probability under brightness
0.88::rotation_0. % Digit 0 correctly predicted with 88% probability under rotation

% Define probabilities for digit 1 under different transformations
0.8::noise_1. % Digit 1 correctly predicted with 80% probability under noise
0.75::brightness_1. % Digit 1 correctly predicted with 75% probability under brightness
0.77::rotation_1. % Digit 1 correctly predicted with 77% probability under rotation

% Define rules for correct prediction under each transformation for digit 0
correct_noise_0 :- noise_0.
correct_brightness_0 :- brightness_0.
correct_rotation_0 :- rotation_0.

% Define rules for correct prediction under each transformation for digit 1
correct_noise_1 :- noise_1.
correct_brightness_1 :- brightness_1.
correct_rotation_1 :- rotation_1.

% Define rules for incorrect prediction under each transformation for digit 0
wrong_noise_0 :- +correct_noise_0.
wrong_brightness_0 :- +correct_brightness_0.
wrong_rotation_0 :- +correct_rotation_0.

% Define rules for incorrect prediction under each transformation for digit 1
wrong_noise_1 :- +correct_noise_1.
wrong_brightness_1 :- +correct_brightness_1.
wrong_rotation_1 :- +correct_rotation_1.

% Define rules for correct prediction of both digits under noise
pair_correct_noise_0_1 :- correct_noise_0, correct_noise_1.
% Define rules for incorrect prediction of both digits under noise
pair_wrong_noise_0_1 :- wrong_noise_0, wrong_noise_1.

% Define global correctness based on either both correct or both incorrect under noise
global_correct_noise_0_1 :- pair_correct_noise_0_1; pair_wrong_noise_0_1.

% Query the global correctness under noise
query(global_correct_noise_0_1).
\end{verbatim}
\end{mdframed}
\captionof{figure}{Problog code snippet for evaluating handwritten digit recognition under noise, brightness, and rotation transformations.}

\textbf{Explanation:} In this scenario, we are interested in the global correctness of recognizing pairs of digits (0 and 1) under different transformations. The ProbLog code models the local robustness probabilities for each transformation and combines them to evaluate the global correctness.

\subsection{Use Case 2: Autonomous Vehicle Perception}

\textbf{Scenario:} An AI system used in autonomous vehicles must reliably detect objects such as vehicles under various weather conditions (rain, sand, fog, and snow). The goal is to evaluate the system's robustness in identifying these objects correctly under these weather conditions.

\begin{mdframed}[leftline=false, rightline=false, topline=true, bottomline=true]
  \scriptsize
  \begin{verbatim}

% Probabilities for Vehicle Detection under Different Weather Conditions
0.75::rain_vehicle. % Vehicle correctly detected with 75% probability under rain
0.55::fog_vehicle. % Vehicle correctly detected with 55% probability under fog
0.7::snow_vehicle. % Vehicle correctly detected with 70% probability under snow

% Correct Detection Rules for Vehicle
correct_rain_vehicle :- rain_vehicle.
correct_fog_vehicle :- fog_vehicle.
correct_snow_vehicle :- snow_vehicle.

% Incorrect Detection Rules for Vehicle
wrong_rain_vehicle :- +correct_rain_vehicle.
wrong_fog_vehicle :- +correct_fog_vehicle.
wrong_snow_vehicle :- +correct_snow_vehicle.

% AND conditions for Vehicle Detection under all weather conditions
global_correct_vehicle_and :- correct_rain_vehicle, correct_fog_vehicle, correct_snow_vehicle.

% OR conditions for Vehicle Detection under any weather condition
global_correct_vehicle_or :- correct_rain_vehicle; correct_fog_vehicle; correct_snow_vehicle.

% Mixed conditions (AND & OR) for Vehicle Detection
global_correct_mixed_vehicle :- correct_rain_vehicle, (correct_fog_vehicle; correct_snow_vehicle).

% Queries for Global Correctness
query(global_correct_vehicle_and).
query(global_correct_vehicle_or).
query(global_correct_mixed_vehicle).
\end{verbatim}
\end{mdframed}
\captionof{figure}{Problog code snippet for evaluating vehicle detection under different weather conditions.}

\textbf{Explanation:} The ProbLog code assesses the global robustness of the AI system in detecting objects (vehicles) under individual and combined weather conditions. This ensures that the system can reliably perform in diverse environmental scenarios.

\begin{table}[h]
  \centering
  \caption{Specification Probabilities (AND) for Vehicle Detection Under Different Weather Conditions \\
  \( P(A \cap B \cap C) = P(A) \times P(B) \times P(C) \)}
  \label{tab:veh_prob_and}
  \resizebox{\textwidth}{!}{%
  \begin{tabular}{|c|c|c|c|}
  \hline
  \textbf{World} & \textbf{Conditions} & \textbf{Probability Expression (AND)} & \textbf{Probability (AND)} \\
  \hline
  $w_1$ & \{rain, fog\} & $0.75 \times 0.55$ & $0.4125$ \\
  $w_2$ & \{rain, snow\} & $0.75 \times 0.7$ & $0.525$ \\
  $w_3$ & \{rain, sand\} & $0.75 \times 0.6$ & $0.45$ \\
  $w_4$ & \{fog, snow\} & $0.55 \times 0.7$ & $0.385$ \\
  $w_5$ & \{fog, sand\} & $0.55 \times 0.6$ & $0.33$ \\
  $w_6$ & \{snow, sand\} & $0.7 \times 0.6$ & $0.42$ \\
  $w_7$ & \{rain, fog, snow\} & $0.75 \times 0.55 \times 0.7$ & $0.28875$ \\
  $w_8$ & \{rain, fog, sand\} & $0.75 \times 0.55 \times 0.6$ & $0.2475$ \\
  $w_9$ & \{rain, snow, sand\} & $0.75 \times 0.7 \times 0.6$ & $0.315$ \\
  $w_{10}$ & \{fog, snow, sand\} & $0.55 \times 0.7 \times 0.6$ & $0.231$ \\
  \hline
  \end{tabular}
  }
\end{table}

\textbf{Explanation:} This table shows the global correctness probabilities for detecting vehicles under different combinations of weather conditions using AND relationships. Each row represents a different combination of weather conditions applied to the detection scenario:
- AND Probability: The probability that the vehicle is correctly detected under all specified weather conditions.

For example, in world $w_1$, the vehicle detection system is subjected to rain and fog, leading to an AND probability of $0.4125$.

\begin{table}[h]
  \centering
  \caption{Specification Probabilities (OR) for Vehicle Detection Under Different Weather Conditions \\
  \( P(A \cup B \cup C) = P(A) + P(B) + P(C) - P(A \cap B) - P(A \cap C) - P(B \cap C) + P(A \cap B \cap C) \)}
  \label{tab:veh_prob_or}
  \resizebox{\textwidth}{!}{%
  \begin{tabular}{|c|c|c|c|}
  \hline
  \textbf{World} & \textbf{Conditions} & \textbf{Probability Expression (OR)} & \textbf{Probability (OR)} \\
  \hline
  $w_1$ & \{rain; fog\} & $0.75 + 0.55 - (0.75 \times 0.55)$ & $0.8875$ \\
  $w_2$ & \{rain; snow\} & $0.75 + 0.7 - (0.75 \times 0.7)$ & $0.925$ \\
  $w_3$ & \{rain; sand\} & $0.75 + 0.6 - (0.75 \times 0.6)$ & $0.9$ \\
  $w_4$ & \{fog; snow\} & $0.55 + 0.7 - (0.55 \times 0.7)$ & $0.835$ \\
  $w_5$ & \{fog; sand\} & $0.55 + 0.6 - (0.55 \times 0.6)$ & $0.82$ \\
  $w_6$ & \{snow; sand\} & $0.7 + 0.6 - (0.7 \times 0.6)$ & $0.88$ \\
  $w_7$ & \{rain; fog; snow\} & $0.75 + 0.55 + 0.7 - (0.75 \times 0.55) - (0.75 \times 0.7) - (0.55 \times 0.7) + (0.75 \times 0.55 \times 0.7)$ & $0.966625$ \\
  $w_8$ & \{rain; fog; sand\} & $0.75 + 0.55 + 0.6 - (0.75 \times 0.55) - (0.75 \times 0.6) - (0.55 \times 0.6) + (0.75 \times 0.55 \times 0.6)$ & $0.95125$ \\
  $w_9$ & \{rain; snow; sand\} & $0.75 + 0.7 + 0.6 - (0.75 \times 0.7) - (0.75 \times 0.6) - (0.7 \times 0.6) + (0.75 \times 0.7 \times 0.6)$ & $0.967$ \\
  $w_{10}$ & \{fog; snow; sand\} & $0.55 + 0.7 + 0.6 - (0.55 \times 0.7) - (0.55 \times 0.6) - (0.7 \times 0.6) + (0.55 \times 0.7 \times 0.6)$ & $0.938$ \\
  \hline
  \end{tabular}
  }
\end{table}

\textbf{Explanation:} This table shows the global correctness probabilities for detecting vehicles under different combinations of weather conditions using OR relationships. Each row represents a different combination of weather conditions applied to the detection scenario:
- OR Probability: The probability that the vehicle is correctly detected under at least one of the specified weather conditions.

For example, in world $w_1$, the vehicle detection system is subjected to rain and fog, leading to an OR probability of $0.8875$.
