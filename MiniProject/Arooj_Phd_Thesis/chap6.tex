\chapter[Conclusion and future work]{Conclusion and future work}
\label{chp:6}
\renewcommand{\bibname}{Conclusion}
\newpage
\section{Conclusion}
In this thesis, we have analyzed issues related to security, computational overhead and conditional privacy in existing vehicular ad dissemination schemes. We have proposed a BC-based ad dissemination scheme that enables conditional anonymity by using pseudonyms instead of ZKPoK, as well as efficient proof verification by using batch verification. The proposed scheme reduces computational cost and enables malicious vehicle detection. Also the vehicle tracing attacks are mitigated by updating the pseudonyms of vehicles after a regular interval. Furthermore, a BC based reputation management system is proposed to promote efficient reputation sharing in VENs. The proposed system initiates with the registration of vehicles through the usage of real and pseudo identities. The registration is done via CA, which also maps the $PIDs$ of the vehicles with their $RIDs$. The $PIDs$ are generated with the help of ECDSA, which ensures conditional anonymity and traceability. In the underlying system, distributed revocation is ensured via SSS algorithm. Moreover, the storage overhead is reduced using IPFS. The reputation data is stored in IPFS while the hashes generated by IPFS are stored in BC. Simulations are performed and the results show the efficacy of the proposed system in terms of computational cost and storage overhead. 18-20\% reduction in computational overhead and 35-40\% reduction in storage overhead are observed when using the proposed system. In the end, the security analysis on the bases of replay attack, 51\% attack and smart contract vulnerabilities prove the model's robustness.

\section{Future work}
In future, we will further improve the performance of our proposed scheme by employing Cuckoo Filters and IOTA Tangle DLT. Moreover, we will develop a mechanism to reuse the pseudonyms that are previously revoked in order to reduce the storage overhead on RSUs.