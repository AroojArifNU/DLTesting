% Adjusting chapter title format for regular (numbered) chapters
\titleformat{\chapter}[display]
  {\normalfont\huge\bfseries\centering}{\chaptertitlename\ \thechapter}{20pt}{\Huge}

% Using similar styling for unnumbered chapters but without "Chapter" prefix
\titleformat{name=\chapter,numberless}
  {\normalfont\huge\bfseries\centering}{}{0pt}{\Huge}

\titlespacing*{\chapter}{0pt}{50pt}{40pt} % Adjust vertical spacing before and after the title

% \chapter{Introduction} % Ensures chapter numbering starts correctly

   \appendix
  \chapter{Glossary}
  \label{gloss}
  % \section{Glossary}



  \textsc{\hyperref[Control-Flow Structure]{\textsc{control-flow structure}}}  refers to the order in which individual statements, instructions, or function calls are executed or evaluated. This structure is typically visualized as a control-flow graph where nodes represent program instructions and edges represent the flow of control between these instructions Section \ref{Control-Flow Structure} of Chapter 1.

  \textsc{\hyperref[Functional coverage]{\textsc{functional coverage}}} used in software testing to determine how well the test cases exercise the functionalities of a system. It measures whether the test cases cover the intended functionality as specified by the requirements or design specifications. Functional coverage focuses on testing the different functionalities and their interactions within the software Section \ref{Functional coverage} of Chapter 1.
  
  \textsc{\hyperref[Branch coverage]{\textsc{branch coverage}}} is a testing metric that measures the percentage of branches or decision points in the code that have been executed by the test cases. It ensures that each possible branch (true/false) of a conditional statement is executed at least once, helping to identify untested paths in the code and ensuring that all logical paths are evaluated Section \ref{Branch coverage} of Chapter 1.


  \textsc{\hyperref[property]{\textsc{property}}} refers to a specific characteristic or feature of the DNN system that is evaluated to ensure its correctness and robustness.
  Section \ref{property} of Chapter 1.
  
  \textsc{\hyperref[formal analysis]{\textsc{formal analysis}}}refers to the use of mathematical and logical methods to verify the correctness and robustness of deep learning models.. Section \ref{formal analysis} of Chapter 1.
 

  \textsc{\hyperref[empirical methods]{\textsc{empirical methods}}}
  approaches that are based on observation, experimentation, and experience rather than purely theoretical analysis Section \ref{empirical methods} of Chapter 1.

  \textsc{\hyperref[Local coverage]{\textsc{local coverage}}}
  refers to evaluating the DNN performance and robustness for each individual class in a dataset separately. This includes assessing  correctness and robustness under various transformations or test cases for each class independently.

  Section \ref{Local coverage} of Chapter 1.

  \textsc{\hyperref[Global coverage]{\textsc{global coverage}}}
  involves assessing the AI system performance and robustness in real-world scenarios where multiple classes interact together. This ensures the model correctness and robustness in dynamic environments with complex class combinations. Section \ref{Global coverage} of Chapter 1.


\textsc{\hyperref[comprehensive]{\textsc{comprehensive}}}  refers to a structured and complete approach designed to cover all necessary aspects and components of a particular system or process. It ensures that every critical element is included and addressed, leaving no gaps. Section \ref{comprehensive} of Chapter 1.
  
\textsc{\hyperref[systematic]{\textsc{systematic}}} refer to an organized approach, often characterized by step-by-step procedures Section \ref{systematic} of Chapter 1.





  % \textsc{\hyperref[neuron coverage]{\uppercase{neuron coverage}}} measures the ratio of neurons activated by test inputs to the total number of neurons in the network.