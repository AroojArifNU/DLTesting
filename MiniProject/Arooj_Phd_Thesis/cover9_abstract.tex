
\addcontentsline{toc}{chapter}{Abstract}
\begin{center}
%{{\fontsize{16}{15} \bf ABSTRACT}\\}
{{\fontsize{16}{15} \bf ABSTRACT
		\\ \vspace{0.3cm} \normalfont Towards Comprehensive Testing Framework for Deep Learning Models
		} \\}
\vspace{0.4cm}
\end{center}
\normalsize

% Data sharing is a fascinating in-vehicle service which provide multiple benefits to the vehicle users in the Vehicular Ad-hoc Networks (VANETs). One of the interesting in-vehicle services is advertisement sharing in VANETs which enable advertisers to market their products and services in the areas of the users interest. With the help of Blockchain (BC) technology, the vehicle users can also participate in the ads dissemination process to gain monetary incentives. However, the existing BC based VANET schemes suffer from privacy, security and efficiency issues. Zero Knowledge Proof of Knowledge (ZKPoK) and certificate-less cryptography are used in the existing schemes to enable fair incentive provision and privacy preservation. These schemes incur high computational cost on the resource constrained vehicles. Moreover, the lack of conditional anonymity in the existing schemes makes the system vulnerable to internal attacker scenario. Furthermore, VANETs require secure and efficient reputation verification mechanism to prevent replay attacks and reduce the storage cost. Additionally, the reliance on a centralized entity for the certificate revocation makes the system wide open to the single point of failure vulnerability. To overcome these issues, a BC based secure, efficient and conditional anonymity enabled scheme is proposed. Elliptic Curve Digital Signature based pseudonym update mechanism is employed to enable conditional anonymity and trace malicious vehicles. InterPlanetary File System is used to efficiently store the vehicles' reputation information and reduce the storage overhead. Moreover, the Shamir Secret Sharing algorithm is used to enable distributed revocation. Security analysis is performed to show that the proposed scheme is secure against multiple known attacks. The simulation results show the effectiveness and practicality of the proposed scheme.



\clearpage
\newpage
